Objetos:🎇

# METODO : es una funcion que se encuentra dentro de un objeto, funcion propia de un objeto o miembro de un objeto

SON LO MISMO PERO LAS DIFERENCIAS SON ↓

# Diferencias entre metodos y funciones: accedemos al metodo através de un objeto en cambio una funcion podemos acceder desde cualquier lado

definición de CLASE: Plano para construír un objetos.

- Las clases comienzan en mayúsculas. ej : Alumno-coder, Usuario, Empleado-home, Programador-developer etc.
- NO es una función, es una clase.
- La clase está compuesta por una función constructora, es el encargado de hacer los seteos iniciales.

Para que los datos no sean accesibles desde afuera por usar clases:

# Get y Set : se encargan de setear variables.

# Getters: Trae los datos, volvemos asignar los parametros con :

get_datos()
{

- console.log("nombre", this.nombre)
- console.log("apellido" this.apellido)
- console.log("Email" this.email)  
}

  alumno_uno.get_datos ();
  llamamos al metodo y se encarga de traer los datos del alumno.

  si después necesito acceder al usuario tendría que invocar la función
  Usuario.get_datos();

# Setters: Setea los datos, evita que asignen y accediendo a las propiedades.

# Arrays 😉

Metodos de array, si queremos acceder a un metodo de un array tenemos que crear un objeto antes
Push,

-Pueden almacenar gran variedad de tipo objetos (String, number) o tambien un array.
-Dinamismo de tipo/equipado: las variables nunca son fijas, cambian de tipo, es decir si ponemos un valor nuevo
no queda ese valor se superpone que es distinto entonces:
NO SE PUEDE ASIGNAR DIRECTO, PORQUE PISO EL ARREGLO
-Son objetos iterables, significa que permite usar distintas estructuras para iterar sobre ellos.  
¿como se accede a los arreglos?
Atraves de sus indices