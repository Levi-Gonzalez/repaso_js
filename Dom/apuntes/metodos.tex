LENGHT : propiedad para saber el largo o longitu del array y tambien cuantos elementos hay

# ej: const clothing = ['shoes', 'shirts', 'socks', 'sweaters']; hay 4 elementos.

Spreed operator: [...nombreDelaVariavble] / {...nombreDeLaVariable} copia un array o un objeto 
trayendo todos sus elementos de la variable

$ForEach$
metodo de orden superior.

$Find$
busca y retorna el primero que encuentra

$Filter$
Retorna el arreglo con los elementos que cumplieron con esa condicion.
Se utiliza para quitar un elemento de un arreglo ej : si nombre !== Basti entonces nos deolvera "Jere" , "Santi"

$Map$ 
crea un nuevo array, con elementos del original cuando apartir de algo queremos tener un nuevo array con modificaciones

$Reduce$
Reduce todo a un solo valor, sirve para calcular las ventas.

$Sort$
Ordena alfabeticamente y modifica el array original.
toma el primer digito del numero y lo ordena pero no tiene en cuenta los siguientes
los ordena de una mala manera la forma correcta es crear esta funcion
para que pueda ordenar los numeros de menos a mayor!!