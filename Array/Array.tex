
//PUSH
listaDeUsuarios.push("Fernando", "Rosana"); // metodo push, agrega al final
console.log(listaDeUsuarios); // RESULTADO DESPUES DEL PUSH

//UNSHIFT
listaDeUsuarios.unshift ( "Blanca") //metodo unshift, agrega al principio

//POP
listaDeUsuarios.pop() //No lo elimina, lo saca y lo retonrna.
console.log("POP" , listaDeUsuarios);

//SHIFT
listaDeUsuarios.shift() //Saca al primero
console.log("SHIFT" , listaDeUsuarios);

//SPLICE
listaDeUsuarios.splice(2,1 , "valor-Agregado-con-splice") //Saca la cantidad que yo quiera atraves de parametros, el 1er parametro nos dice DESDE y el 2do CUANTOS.
//Y desde esa posicion tambien podemos agregar un nuevo valor
console.log("SPLICE", listaDeUsuarios);

//JOIN
let resultado = listaDeUsuarios.join (" | ") //transforma todos los elemtos del array en un string y su funcion es con que caracterer separo cada string
console.log("JOIN:" , resultado);
console.log("Join:", listaDeUsuarios.join (" \*\*\* "));

//CONCAT
let colores = ["Azul", "Amarillo" , "Rojo"] //fusiona 2 arrays en uno nuevo, es decir habría un 3er array nuevo, el que va ultimo se agrega a lo ultimo.
let nuevaLista = listaDeUsuarios.concat(colores)
console.log("CONCAT:" , nuevaLista);

//SLICE
let corte = nuevaLista.slice (2,4) //Corta desde el numero indicado Hasta el otro, el ULTIMO NO ESTA INCLUIDO, por lo que es el anterior al ultimo
//luego retorna los que corto mostrandolos en consola
console.log("SLICE:" , corte);

//iOF
//NOs retorna el indice de un array, tenemos que poner el elemento que buscamos en eel array si no existe nos retorna -1
console.log("iOF", nuevaLista.indexOf("Basti")); //0
console.log("INDEXOF", nuevaLista.indexOf("Azul")); //7
console.log("INDEXOF", nuevaLista.indexOf("verde")); //-1 porque no existe

//COMBINACION DE: INDEXOF y SPLICE
let borrar = nuevaLista.indexOf ("Fernando") //lo ubicamos en el array
nuevaLista.splice (borrar, 1) //eliminamos elementos puntuales, si no especificamos la cantidad borra desde ahí para atras
console.log("INDEXOF y SPLICE:" , nuevaLista); //Esta es la manera de borrar elementos de un array! 😉

//INCLUDES
let includes = listaDeUsuarios.includes("rita") //Si existe en el array devuelve TRUE, si no existe FALSE
console.log(includes);

//REVERSE
nuevaLista.reverse() //Invierte o modifica el array original
console.log(nuevaLista);

FOR...OF
RECORRE EL ARRAY ejecutando un bloque de codigo por cada elemento del OBJETO
Si tiene 20 alumnos → 20 vueltas, 50 usuarios → 50 vueltas, 1000 personajes → 1000 vueltas
Es decir, los separa cada objeto por separado

# 7

/\*ForEach → metodo de orden superior.

let numeros = [1,2,3,4,5,6,7,8,9]
numeros .forEach(num => {
console.log("El numero es" + " " + num);
console.log("El doble del numero es" + " " + num*2);
});
*/

/\* Find → busca y retorna el primero que encuentra

const usuarios = [
{nombre: "Basti", apellido: "Gonzalez", edad:2, DNI: 58205741},
{nombre: "Jere", apellido: "Gonzalez", edad:15,DNI: 123456789},
{nombre: "Santi", apellido: "Gonzalez", edad:13, DNI:987456321}
]

const buscarNombre = (usuarios) =>{
return usuarios.DNI == 58205741
}
let resultado_find = usuarios.find(buscarNombre)
console.log(resultado_find);
\*/

/\*Filter → Retorna el arreglo con los elementos que cumplieron con esa condicion.
Se utiliza para quitar un elemento de un arreglo ej : si nombre !== Basti entonces nos deolvera "Jere" , "Santi"
const usuarios = [
{nombre: "Basti", apellido: "Gonzalez", edad:2, DNI: 58205741},
{nombre: "Jere", apellido: "Gonzalez", edad:15,DNI: 123456789},
{nombre: "Santi", apellido: "Gonzalez", edad:13, DNI:987456321}
]

const mayor_de_edad = (usuarios) =>{

    return usuarios.edad >= 10

}

let resultado_filter = usuarios.filter (mayor_de_edad)
console.log(resultado_filter);
\*/

let precioTotal = ventas.reduce (valorTotal, 0)
console.log("el valor total es:" , precioTotal);

En este ejercicio construiremos una herramienta que permita que diferentes personas puedan
llevar cuentas individuales sobre algo que deseen contabilizar, al mismo tiempo que nos brinde una contabilidad
general del total contado. Para ello:
Definir la clase Contador.
Cada instancia de contador debe ser identificada con el nombre de la persona responsable de ese conteo.
Cada instancia inicia su cuenta individual en cero.
La clase en sí misma posee un valor estático con el que lleva la cuenta de todo lo contado
// por sus instancias, el cual también inicia en cero.


//nombre de referencia ↓  
for (const random of personajesFavoritos) // ← nombre de la lista/array que va a recorrer
{
console.log(random);
}
// FOR OF recorre y separa los objetos por separados.
