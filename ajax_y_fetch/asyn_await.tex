Async await: $Ahorra los then$
Async = es una funcion.

¿Como darnos cuenta cuando un método es async?
Es async cuando lo manejamos através de una función.

await:
# Solamente funciona con funciones Async
# Quiere decir como : espera,  porque esto necesita un tiempo para responder.

%¿Cuando sabemos que hay que usar await?
- Cuando guardamos una respuesta en una variable.
%¿Como utilizar await?
- La función tiene que ser Async

%¿Cómo defino una funcion Async?
- Agregandole Async antes del function.

%¿Pero como hago que funcione?
$ Tiene que estar adentro de una funcion async$

{Crear un setTimeOut}
Hay que importarlo con el modulo de NODE y se llama UTIL.

ej: 

const respuesta = async function (){
    let resultado = await fetch ("https://pokeapi.co/api/v2/berry/10")
    console.log (resultado)
}

cuando se ejecute esta funcion async y lea el https va a decir: "espera que estoy esperando esta respuesta del pro ceso async"
una vez que este proceso async termine el console se ejecutara.

const util = require ('util')
const sleep = util.promisify(setTimeOut)
este modulo nos trae a 'promisify' convierte a las funciones que se manejan con callbacks a funciones que se manejan con promesas o async await...
Con esto decimos que queremos convertir a 'setTimeOut' de un callback a una promesa. Luego, la guardamos a una variable.
Entonces este código tiene la misma funcion que 'setTimeOut' pero lo puedo ejecutar através asyn await.
Cada vez que la querramos usar tenemos que poner : $ await sleep ()  $
adentro va como parametro el tiempo que quiero que espere: $  await sleep(2000)  $

Quedaría asi:
const util = require('util');
const sleep = util.promisify(setTimeOut);

module.exports = {

    async taksOne(){
        await sleep (4000)
        return "One love"
    },
    async taskTwo(){
        sleep(2000)
        return "Take away"
    }


}