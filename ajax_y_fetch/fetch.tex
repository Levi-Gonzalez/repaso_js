Fetch:
* Me permite hacer paticiones http asincrónicas
* Trabaja directamente con promesas
* Cuando hagamos una petición con fetch lo que nos ${DEVUELVE}$ sera una promesa de esa api que nos este respondiendo

fetch(url, config)  
1) El primer parámetro que pondremos es el url donde queramos hacer la petición.
2) Podemos especificar la configuración por ej: si quieramos especificar el métodos , header, body.
si quiero especificar algo más se puede poner todo en ese mismo fetch.   
° Como es una promesa hay que usar un then para consumirla.
${DEVUELVE UNA PROMESA}$ se resuelve como una respuesta pero no se puede consumir en ese momento, entonces casi siempre cuando querramos barrer 
nos quedara un anidamiento de then.
+ Para consumirlas hay que llamar al método de las promesas que es ".json" y que hace este método? el body de la respuesta lo convierte en JSON
y ahi lo capturo en otro then porque lo que estoy devolviendo es otra promesa más.

ej: 
    ________________________________________________________________
    |        fetch("https://jsonplaceholder.typicode.com/posts")     |       
    |        .then((resp) => resp.json ())                           |     
    |        .then((data) => {                                       |     
    |            console.log(data)                                   |                                                        
    |        }                                                       |   
    |________________________________________________________________|                                                                         
    
    2. manera de escribirlo: si estamos poniendo un valor se supone que ese valor se esta retornando.
    _______________________________________________________________
    |fetch("https://jsonplaceholder.typicode.com/posts" )     |     |
    |.then(function (resp) { return resp.json ()})                  | 
    |.then(data => console.log(data));                              |             
    |_______________________________________________________________|

Then = función consumidora = quiere decir que estamos pasando un callback
¿Que pasa en esta línea?     →   .then((resp) => resp.json ())

(resp) => resp.json () %Lo que llega acá es una promesa que se resuelve en un objeto respuesta.%
ese objeto respuesta $resp.json()$ %Lo que estamos haciendo acá es convertir el body a donde viene todda la info de la respuesta a json y eso retorna%
Lo que hace ese .JSON agara
_____________________________________________
fetch('https://fakestoreapi.com/products/1')
            .then(res=>res.json())
            .then(data=>console.log(data))

¿Pórque no funciona esto?
Porque fetch hace una petición http, es decir manda una petición al servidor y no tenemos un servidor local por ende falla.
Si tenes un servidor local esto funcionaria, la extension "Live server" o levantar con un servidor con un express.




#RESUMEN: 
Fetch retorna una promesa
Recibimos esa respuesta
al body de la respuesta lo convertimos a json para hacerlo consumible..
y la enviamos al otro then. 
los then se enganchan en otro then, then, then [...].

$DATO IMPORTANTE:$
° Cuando  trae el error 404  en el fetch nos mostrata que la operacion se realizo bien
