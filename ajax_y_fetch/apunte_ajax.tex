¿Que es Ajax?
* Es una forma de enviar o recibir una información sin tener que estar recargando la página.
* Através de ajax podemos traer información para completar.
* Es un conjunto de técnicad  que permite que las aplicaiones web funcionen de manera asincrónica. es decir, parte de lo que es js asincrónico.

Peticiones http : hypertext - transfer - protocol. 
Cuando emitimos una orden al navegador, hace una petición (request).
cuando el servidor da:
200: va correcta ✔
400 (...) : quiere decir que el servidor esta funcionando, pero no encuentra lo que el usuario esta buscando. | el URL esta mal | el sitio no existe más.
404: el archivo no se encontró 🤔
500: problema en el servidor directo, quiere decir que no se pudo conectar con el servidor. | esta caido ❌| esta "palmado" ❌.

la peticiones http estan compuestas por:
- url donde se va a pedir ese recurso. 
- un método (get, post, put, delete y más pero eso son los más importantes).
- headers
- body
- parametros (query params o url params )
- el método por default es el get.
________________________________________________________________________________________________________________________________________________________________

Método: es la forma con la que interactuamos con el servidor.
GET: Cuando hacemos una peticion, trae información/trae datos(www.google.com).
POST: Manda la data. Ej: en el backend cuando se llena un formulario de alta, se usa el método "post".  
PUT: Similar al post porque envia datos tambien, pero esta pensado para crear, principalmente modificar. Manda una petición de modificación.
DELETE: Sirve para elminar algún recurso en el servidor.

POST y PUT: 
Son las unicas peticiones que van acompañada de un body (cuerpo de la petición/ request)
¿Por qué tienen un cuerpo?
Porque necesitan enviar data para modificar y hacer altas.
________________________________________________________________________________________________________________________________________________________________

Headers: Las cabeceras (header) http, nos permite hacer determinadas especificaciones para que cuando yo mande o reciba la petición salga correcta.
Parámetros: Son los que estan en la URL, los  #Query params o #URL params               
    
    
                                                                            ______

                                                                            30:00
                                                                            _______

